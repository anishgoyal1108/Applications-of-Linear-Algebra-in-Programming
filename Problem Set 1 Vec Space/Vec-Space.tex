% Options for packages loaded elsewhere
\PassOptionsToPackage{unicode}{hyperref}
\PassOptionsToPackage{hyphens}{url}
\PassOptionsToPackage{dvipsnames,svgnames,x11names}{xcolor}
%
\documentclass[
  letterpaper,
  DIV=11,
  numbers=noendperiod]{scrartcl}

\usepackage{amsmath,amssymb}
\usepackage{lmodern}
\usepackage{iftex}
\ifPDFTeX
  \usepackage[T1]{fontenc}
  \usepackage[utf8]{inputenc}
  \usepackage{textcomp} % provide euro and other symbols
\else % if luatex or xetex
  \usepackage{unicode-math}
  \defaultfontfeatures{Scale=MatchLowercase}
  \defaultfontfeatures[\rmfamily]{Ligatures=TeX,Scale=1}
\fi
% Use upquote if available, for straight quotes in verbatim environments
\IfFileExists{upquote.sty}{\usepackage{upquote}}{}
\IfFileExists{microtype.sty}{% use microtype if available
  \usepackage[]{microtype}
  \UseMicrotypeSet[protrusion]{basicmath} % disable protrusion for tt fonts
}{}
\makeatletter
\@ifundefined{KOMAClassName}{% if non-KOMA class
  \IfFileExists{parskip.sty}{%
    \usepackage{parskip}
  }{% else
    \setlength{\parindent}{0pt}
    \setlength{\parskip}{6pt plus 2pt minus 1pt}}
}{% if KOMA class
  \KOMAoptions{parskip=half}}
\makeatother
\usepackage{xcolor}
\setlength{\emergencystretch}{3em} % prevent overfull lines
\setcounter{secnumdepth}{-\maxdimen} % remove section numbering
% Make \paragraph and \subparagraph free-standing
\ifx\paragraph\undefined\else
  \let\oldparagraph\paragraph
  \renewcommand{\paragraph}[1]{\oldparagraph{#1}\mbox{}}
\fi
\ifx\subparagraph\undefined\else
  \let\oldsubparagraph\subparagraph
  \renewcommand{\subparagraph}[1]{\oldsubparagraph{#1}\mbox{}}
\fi

\usepackage{color}
\usepackage{fancyvrb}
\newcommand{\VerbBar}{|}
\newcommand{\VERB}{\Verb[commandchars=\\\{\}]}
\DefineVerbatimEnvironment{Highlighting}{Verbatim}{commandchars=\\\{\}}
% Add ',fontsize=\small' for more characters per line
\usepackage{framed}
\definecolor{shadecolor}{RGB}{241,243,245}
\newenvironment{Shaded}{\begin{snugshade}}{\end{snugshade}}
\newcommand{\AlertTok}[1]{\textcolor[rgb]{0.68,0.00,0.00}{#1}}
\newcommand{\AnnotationTok}[1]{\textcolor[rgb]{0.37,0.37,0.37}{#1}}
\newcommand{\AttributeTok}[1]{\textcolor[rgb]{0.40,0.45,0.13}{#1}}
\newcommand{\BaseNTok}[1]{\textcolor[rgb]{0.68,0.00,0.00}{#1}}
\newcommand{\BuiltInTok}[1]{\textcolor[rgb]{0.00,0.23,0.31}{#1}}
\newcommand{\CharTok}[1]{\textcolor[rgb]{0.13,0.47,0.30}{#1}}
\newcommand{\CommentTok}[1]{\textcolor[rgb]{0.37,0.37,0.37}{#1}}
\newcommand{\CommentVarTok}[1]{\textcolor[rgb]{0.37,0.37,0.37}{\textit{#1}}}
\newcommand{\ConstantTok}[1]{\textcolor[rgb]{0.56,0.35,0.01}{#1}}
\newcommand{\ControlFlowTok}[1]{\textcolor[rgb]{0.00,0.23,0.31}{#1}}
\newcommand{\DataTypeTok}[1]{\textcolor[rgb]{0.68,0.00,0.00}{#1}}
\newcommand{\DecValTok}[1]{\textcolor[rgb]{0.68,0.00,0.00}{#1}}
\newcommand{\DocumentationTok}[1]{\textcolor[rgb]{0.37,0.37,0.37}{\textit{#1}}}
\newcommand{\ErrorTok}[1]{\textcolor[rgb]{0.68,0.00,0.00}{#1}}
\newcommand{\ExtensionTok}[1]{\textcolor[rgb]{0.00,0.23,0.31}{#1}}
\newcommand{\FloatTok}[1]{\textcolor[rgb]{0.68,0.00,0.00}{#1}}
\newcommand{\FunctionTok}[1]{\textcolor[rgb]{0.28,0.35,0.67}{#1}}
\newcommand{\ImportTok}[1]{\textcolor[rgb]{0.00,0.46,0.62}{#1}}
\newcommand{\InformationTok}[1]{\textcolor[rgb]{0.37,0.37,0.37}{#1}}
\newcommand{\KeywordTok}[1]{\textcolor[rgb]{0.00,0.23,0.31}{#1}}
\newcommand{\NormalTok}[1]{\textcolor[rgb]{0.00,0.23,0.31}{#1}}
\newcommand{\OperatorTok}[1]{\textcolor[rgb]{0.37,0.37,0.37}{#1}}
\newcommand{\OtherTok}[1]{\textcolor[rgb]{0.00,0.23,0.31}{#1}}
\newcommand{\PreprocessorTok}[1]{\textcolor[rgb]{0.68,0.00,0.00}{#1}}
\newcommand{\RegionMarkerTok}[1]{\textcolor[rgb]{0.00,0.23,0.31}{#1}}
\newcommand{\SpecialCharTok}[1]{\textcolor[rgb]{0.37,0.37,0.37}{#1}}
\newcommand{\SpecialStringTok}[1]{\textcolor[rgb]{0.13,0.47,0.30}{#1}}
\newcommand{\StringTok}[1]{\textcolor[rgb]{0.13,0.47,0.30}{#1}}
\newcommand{\VariableTok}[1]{\textcolor[rgb]{0.07,0.07,0.07}{#1}}
\newcommand{\VerbatimStringTok}[1]{\textcolor[rgb]{0.13,0.47,0.30}{#1}}
\newcommand{\WarningTok}[1]{\textcolor[rgb]{0.37,0.37,0.37}{\textit{#1}}}

\providecommand{\tightlist}{%
  \setlength{\itemsep}{0pt}\setlength{\parskip}{0pt}}\usepackage{longtable,booktabs,array}
\usepackage{calc} % for calculating minipage widths
% Correct order of tables after \paragraph or \subparagraph
\usepackage{etoolbox}
\makeatletter
\patchcmd\longtable{\par}{\if@noskipsec\mbox{}\fi\par}{}{}
\makeatother
% Allow footnotes in longtable head/foot
\IfFileExists{footnotehyper.sty}{\usepackage{footnotehyper}}{\usepackage{footnote}}
\makesavenoteenv{longtable}
\usepackage{graphicx}
\makeatletter
\def\maxwidth{\ifdim\Gin@nat@width>\linewidth\linewidth\else\Gin@nat@width\fi}
\def\maxheight{\ifdim\Gin@nat@height>\textheight\textheight\else\Gin@nat@height\fi}
\makeatother
% Scale images if necessary, so that they will not overflow the page
% margins by default, and it is still possible to overwrite the defaults
% using explicit options in \includegraphics[width, height, ...]{}
\setkeys{Gin}{width=\maxwidth,height=\maxheight,keepaspectratio}
% Set default figure placement to htbp
\makeatletter
\def\fps@figure{htbp}
\makeatother

\usepackage{amsmath, xparse}
\usepackage{fancyvrb, fvextra}
\usepackage{bm}
\usepackage{svg}
\usepackage{listings}
\DefineVerbatimEnvironment{Highlighting}{Verbatim}{breaklines,commandchars=\\\{\}}
\lstset{basicstyle=\ttfamily\footnotesize,breaklines=true}
\KOMAoption{captions}{tableheading}
\makeatletter
\makeatother
\makeatletter
\makeatother
\makeatletter
\@ifpackageloaded{caption}{}{\usepackage{caption}}
\AtBeginDocument{%
\ifdefined\contentsname
  \renewcommand*\contentsname{Table of contents}
\else
  \newcommand\contentsname{Table of contents}
\fi
\ifdefined\listfigurename
  \renewcommand*\listfigurename{List of Figures}
\else
  \newcommand\listfigurename{List of Figures}
\fi
\ifdefined\listtablename
  \renewcommand*\listtablename{List of Tables}
\else
  \newcommand\listtablename{List of Tables}
\fi
\ifdefined\figurename
  \renewcommand*\figurename{Figure}
\else
  \newcommand\figurename{Figure}
\fi
\ifdefined\tablename
  \renewcommand*\tablename{Table}
\else
  \newcommand\tablename{Table}
\fi
}
\@ifpackageloaded{float}{}{\usepackage{float}}
\floatstyle{ruled}
\@ifundefined{c@chapter}{\newfloat{codelisting}{h}{lop}}{\newfloat{codelisting}{h}{lop}[chapter]}
\floatname{codelisting}{Listing}
\newcommand*\listoflistings{\listof{codelisting}{List of Listings}}
\makeatother
\makeatletter
\@ifpackageloaded{caption}{}{\usepackage{caption}}
\@ifpackageloaded{subcaption}{}{\usepackage{subcaption}}
\makeatother
\makeatletter
\@ifpackageloaded{tcolorbox}{}{\usepackage[many]{tcolorbox}}
\makeatother
\makeatletter
\@ifundefined{shadecolor}{\definecolor{shadecolor}{rgb}{.97, .97, .97}}
\makeatother
\makeatletter
\makeatother
\ifLuaTeX
  \usepackage{selnolig}  % disable illegal ligatures
\fi
\IfFileExists{bookmark.sty}{\usepackage{bookmark}}{\usepackage{hyperref}}
\IfFileExists{xurl.sty}{\usepackage{xurl}}{} % add URL line breaks if available
\urlstyle{same} % disable monospaced font for URLs
\hypersetup{
  colorlinks=true,
  linkcolor={blue},
  filecolor={Maroon},
  citecolor={Blue},
  urlcolor={Blue},
  pdfcreator={LaTeX via pandoc}}

\author{}
\date{}

\begin{document}
\begin{titlepage}

    \newcommand{\HRule}{\rule{\linewidth}{0.5mm}}
    
    \center
    
    \textsc{\LARGE GSMST }\\[0.3cm]
    \textsc{\Large Applications of Linear Algebra }\\[0.3cm]
    \textsc{\Large in Programming}\\[0.5cm]
    
    \HRule \\[0.4cm]
    { \huge \bfseries Chapter 3 Assignment}\\[0.03cm]
    \HRule \\[1.5cm]
    
    \begin{minipage}{0.4\textwidth}
    \begin{flushleft} \large
    \emph{Submitted By:}\\
    Anish Goyal \\4th Period
    \end{flushleft}
    \end{minipage}
    ~
    \begin{minipage}{0.4\textwidth}
    \begin{flushright} \large
    \emph{Submitted To:} \\
    Mrs. Denise Stiffler\\Educator
    \end{flushright}
    \end{minipage}\\[1cm]
    
    {\large April 11, 2023}\\[1cm]
    
    \includegraphics{logo.png}\\[1cm]
    \vfill
    \end{titlepage}
\newpage2

\ifdefined\Shaded\renewenvironment{Shaded}{\begin{tcolorbox}[frame hidden, borderline west={3pt}{0pt}{shadecolor}, boxrule=0pt, breakable, interior hidden, sharp corners, enhanced]}{\end{tcolorbox}}\fi

\renewcommand*\contentsname{Table of contents}
{
\hypersetup{linkcolor=}
\setcounter{tocdepth}{4}
\tableofcontents
}
\newpage{}

\hypertarget{problems}{%
\section{3.8 Problems}\label{problems}}

\hypertarget{problem-3.8.1}{%
\subsection{Problem 3.8.1}\label{problem-3.8.1}}

\begin{enumerate}
\def\labelenumi{\arabic{enumi}.}
\item
  Write and test a procedure \texttt{vec\_select} using a comprehension
  for the following computational problem:

  \begin{itemize}
  \tightlist
  \item
    \emph{input:} a list \texttt{veclist} of vectors over the same
    domain, and an element \(k\) of the domain
  \item
    \emph{output:} the sublist of \texttt{veclist} consisting of the
    vectors \texttt{v} in \texttt{veclist} where \texttt{v{[}k{]}} is
    zero
  \end{itemize}
\item
  Write and test a procedure \texttt{vec\_sum} using the built-in
  procedure \texttt{sum(.)} for the following:

  \begin{itemize}
  \tightlist
  \item
    \emph{input:} a list \texttt{veclist} of vectors, and a set
    \texttt{D} that is the common domain of these vectors
  \item
    \emph{output:} the vector sum of the vectors in \texttt{veclist}
  \end{itemize}

  Your procedure must work even if \texttt{veclist} has length 0.\\
  \emph{Hint:} Recall from the Python lab that \texttt{sum(.)}
  optionally takes a second argument, which is the element to start the
  sum with. This can be a vector.\\
  \emph{Disclaimer}: The \texttt{Vec} class is defined in such a way
  that, for a vector \texttt{v}, the expression \texttt{0\ +\ v}
  evaluates to \texttt{v}. This was done precisely so that
  \texttt{sum({[}v1,v2,...\ vk{]})} will correctly evaluate to the sum
  of the vectors when the number of vectors is nonzero. However, this
  won't work when the number of vectors is zero.
\item
  Put your procedures together to obtain a procedure
  \texttt{vec\_select\_sum} for the following:

  \begin{itemize}
  \tightlist
  \item
    \emph{input}: a set \texttt{D}, a list \texttt{veclist} of vectors
    within domain \texttt{D}, and an element \texttt{k} of the domain
  \item
    \emph{output}: the sum of all vectors \texttt{v} in \texttt{veclist}
    where \texttt{v{[}k{]}} is zero
  \end{itemize}
\end{enumerate}

\newpage{}

\begin{Shaded}
\begin{Highlighting}[numbers=left,,]
\ImportTok{from}\NormalTok{ vec }\ImportTok{import}\NormalTok{ Vec}
\ImportTok{from}\NormalTok{ vecutil }\ImportTok{import}\NormalTok{ list2vec}

\KeywordTok{def}\NormalTok{ vec\_select(veclist, k):}
    \ControlFlowTok{return}\NormalTok{ [v }\ControlFlowTok{for}\NormalTok{ v }\KeywordTok{in}\NormalTok{ veclist }\ControlFlowTok{if}\NormalTok{ v[k] }\OperatorTok{==} \DecValTok{0}\NormalTok{]}

\NormalTok{veclist }\OperatorTok{=}\NormalTok{ [list2vec(}\BuiltInTok{range}\NormalTok{(}\DecValTok{15}\NormalTok{)), list2vec([}\DecValTok{1}\NormalTok{, }\DecValTok{2}\NormalTok{, }\DecValTok{3}\NormalTok{]), list2vec([}\DecValTok{0}\NormalTok{, }\DecValTok{0}\NormalTok{, }\DecValTok{1}\NormalTok{])]}
\BuiltInTok{print}\NormalTok{(vec\_select(veclist, }\DecValTok{0}\NormalTok{))}
\BuiltInTok{print}\NormalTok{(vec\_select(veclist, }\DecValTok{1}\NormalTok{))}
\BuiltInTok{print}\NormalTok{(vec\_select(veclist, }\DecValTok{2}\NormalTok{))}
\end{Highlighting}
\end{Shaded}

\begin{lstlisting}
[Vec({0, 1, 2, 3, 4, 5, 6, 7, 8, 9, 10, 11, 12, 13, 14},{0: 0, 1: 1, 2: 2, 3: 3, 4: 4, 5: 5, 6: 6, 7: 7, 8: 8, 9: 9, 10: 10, 11: 11, 12: 12, 13: 13, 14: 14}), Vec({0, 1, 2},{0: 0, 1: 0, 2: 1})]
[Vec({0, 1, 2},{0: 0, 1: 0, 2: 1})]
[]
\end{lstlisting}

\begin{Shaded}
\begin{Highlighting}[numbers=left,,]
\KeywordTok{def}\NormalTok{ vec\_sum(veclist, D):}
    \ControlFlowTok{return} \BuiltInTok{sum}\NormalTok{([Vec(D, v.f) }\ControlFlowTok{for}\NormalTok{ v }\KeywordTok{in}\NormalTok{ veclist], Vec(D, \{\}))}

\BuiltInTok{print}\NormalTok{(vec\_sum([], \{}\StringTok{\textquotesingle{}a\textquotesingle{}}\NormalTok{, }\StringTok{\textquotesingle{}b\textquotesingle{}}\NormalTok{, }\StringTok{\textquotesingle{}c\textquotesingle{}}\NormalTok{\}))}
\BuiltInTok{print}\NormalTok{(vec\_sum([list2vec([}\DecValTok{1}\NormalTok{, }\DecValTok{2}\NormalTok{, }\DecValTok{3}\NormalTok{]), list2vec([}\DecValTok{4}\NormalTok{, }\DecValTok{5}\NormalTok{, }\DecValTok{6}\NormalTok{]), list2vec([}\DecValTok{7}\NormalTok{, }\DecValTok{8}\NormalTok{, }\DecValTok{9}\NormalTok{])], \{}\StringTok{\textquotesingle{}x\textquotesingle{}}\NormalTok{, }\StringTok{\textquotesingle{}y\textquotesingle{}}\NormalTok{, }\StringTok{\textquotesingle{}z\textquotesingle{}}\NormalTok{\}))}
\end{Highlighting}
\end{Shaded}

\begin{lstlisting}

 a b c
------
 0 0 0

 x y z
------
 0 0 0
\end{lstlisting}

\begin{Shaded}
\begin{Highlighting}[numbers=left,,]
\KeywordTok{def}\NormalTok{ vec\_select\_sum(D, veclist, k):}
    \ControlFlowTok{return}\NormalTok{ vec\_sum(vec\_select (veclist, k), D)}

\BuiltInTok{print}\NormalTok{(vec\_select\_sum(veclist[}\DecValTok{0}\NormalTok{].D, veclist, }\DecValTok{0}\NormalTok{))}
\end{Highlighting}
\end{Shaded}

\begin{lstlisting}

 0 1 10 11 12 13 14 2 3 4 5 6 7 8 9
-----------------------------------
 0 1 10 11 12 13 14 3 3 4 5 6 7 8 9
\end{lstlisting}

\newpage{}

\hypertarget{problem-3.8.2}{%
\subsection{Problem 3.8.2}\label{problem-3.8.2}}

Write and test a procedure \texttt{scale\_vecs(vecdict)} for the
following:

\begin{itemize}
\tightlist
\item
  \emph{input}: a dictionary vector mapping positive numbers to vectors
  (instances of \texttt{Vec})
\item
  \emph{output}: a list of vectors, one for each item in
  \texttt{vecdict}. If \texttt{vecdict} contains a key \emph{k} mapping
  to vector \(\bm{v}\), the output should contain the vector
  \((1/k)\bm{v}\)
\end{itemize}

\begin{Shaded}
\begin{Highlighting}[numbers=left,,]
\KeywordTok{def}\NormalTok{ scale\_vecs(vecdict):}
    \ControlFlowTok{return}\NormalTok{ [(}\DecValTok{1}\OperatorTok{/}\NormalTok{k)}\OperatorTok{*}\NormalTok{v }\ControlFlowTok{for}\NormalTok{ k, v }\KeywordTok{in}\NormalTok{ vecdict.items()]}

\NormalTok{v1 }\OperatorTok{=}\NormalTok{ Vec(\{}\StringTok{\textquotesingle{}x\textquotesingle{}}\NormalTok{, }\StringTok{\textquotesingle{}y\textquotesingle{}}\NormalTok{, }\StringTok{\textquotesingle{}z\textquotesingle{}}\NormalTok{\}, \{}\StringTok{\textquotesingle{}x\textquotesingle{}}\NormalTok{: }\DecValTok{1}\NormalTok{, }\StringTok{\textquotesingle{}y\textquotesingle{}}\NormalTok{: }\DecValTok{2}\NormalTok{, }\StringTok{\textquotesingle{}z\textquotesingle{}}\NormalTok{: }\DecValTok{3}\NormalTok{\})}
\NormalTok{v2 }\OperatorTok{=}\NormalTok{ Vec(\{}\StringTok{\textquotesingle{}x\textquotesingle{}}\NormalTok{, }\StringTok{\textquotesingle{}y\textquotesingle{}}\NormalTok{, }\StringTok{\textquotesingle{}z\textquotesingle{}}\NormalTok{\}, \{}\StringTok{\textquotesingle{}x\textquotesingle{}}\NormalTok{: }\DecValTok{4}\NormalTok{, }\StringTok{\textquotesingle{}y\textquotesingle{}}\NormalTok{: }\DecValTok{5}\NormalTok{, }\StringTok{\textquotesingle{}z\textquotesingle{}}\NormalTok{: }\DecValTok{6}\NormalTok{\})}
\NormalTok{v3 }\OperatorTok{=}\NormalTok{ Vec(\{}\StringTok{\textquotesingle{}x\textquotesingle{}}\NormalTok{, }\StringTok{\textquotesingle{}y\textquotesingle{}}\NormalTok{, }\StringTok{\textquotesingle{}z\textquotesingle{}}\NormalTok{\}, \{}\StringTok{\textquotesingle{}x\textquotesingle{}}\NormalTok{: }\DecValTok{7}\NormalTok{, }\StringTok{\textquotesingle{}y\textquotesingle{}}\NormalTok{: }\DecValTok{8}\NormalTok{, }\StringTok{\textquotesingle{}z\textquotesingle{}}\NormalTok{: }\DecValTok{9}\NormalTok{\})}

\NormalTok{vecdict }\OperatorTok{=}\NormalTok{ \{}\DecValTok{2}\NormalTok{: v1, }\DecValTok{3}\NormalTok{: v2, }\DecValTok{4}\NormalTok{: v3\}}

\BuiltInTok{print}\NormalTok{([v }\ControlFlowTok{for}\NormalTok{ v }\KeywordTok{in}\NormalTok{ scale\_vecs(vecdict)]) }
\end{Highlighting}
\end{Shaded}

\begin{lstlisting}
[Vec({'z', 'x', 'y'},{'z': 1.5, 'x': 0.5, 'y': 1.0}), Vec({'z', 'x', 'y'},{'z': 2.0, 'x': 1.3333333333333333, 'y': 1.6666666666666665}), Vec({'z', 'x', 'y'},{'z': 2.25, 'x': 1.75, 'y': 2.0})]
\end{lstlisting}

\newpage{}

\hypertarget{problem-3.8.3}{%
\subsection{Problem 3.8.3}\label{problem-3.8.3}}

Write a procedure \texttt{GF2\_span} with the following specs:

\begin{itemize}
\tightlist
\item
  \emph{input}: a set \emph{D} of labels and a list \emph{L} of vectors
  over \emph{GF(2)} with label-set \emph{D}
\item
  \emph{output}: the list of all linear combinations of the vectors in
  \emph{L}
\end{itemize}

(Hint: use a loop (or recursion) and a comprehension. Be sure to test
your procedure on examples where \emph{L} is an empty list.)

\begin{Shaded}
\begin{Highlighting}[numbers=left,,]
\ImportTok{from}\NormalTok{ itertools }\ImportTok{import}\NormalTok{ product}

\KeywordTok{def}\NormalTok{ GF2\_span(D, L):}
    \ControlFlowTok{if} \KeywordTok{not}\NormalTok{ L:}
        \ControlFlowTok{return}\NormalTok{ [Vec(D, \{\})]}
\NormalTok{    result }\OperatorTok{=}\NormalTok{ []}
    \ControlFlowTok{for}\NormalTok{ coeffs }\KeywordTok{in}\NormalTok{ product([}\DecValTok{0}\NormalTok{,}\DecValTok{1}\NormalTok{], repeat}\OperatorTok{=}\BuiltInTok{len}\NormalTok{(L)):}
\NormalTok{        combination }\OperatorTok{=}\NormalTok{ Vec(D, \{\})}
        \ControlFlowTok{for}\NormalTok{ i, v }\KeywordTok{in} \BuiltInTok{enumerate}\NormalTok{(L):}
\NormalTok{            combination }\OperatorTok{+=}\NormalTok{ coeffs[i]}\OperatorTok{*}\NormalTok{v}
\NormalTok{        result.append(combination)}
    \ControlFlowTok{return}\NormalTok{ result}

\NormalTok{D }\OperatorTok{=}\NormalTok{ \{}\StringTok{\textquotesingle{}a\textquotesingle{}}\NormalTok{, }\StringTok{\textquotesingle{}b\textquotesingle{}}\NormalTok{, }\StringTok{\textquotesingle{}c\textquotesingle{}}\NormalTok{\}}
\NormalTok{L }\OperatorTok{=}\NormalTok{ [Vec(D, \{}\StringTok{\textquotesingle{}a\textquotesingle{}}\NormalTok{: }\DecValTok{1}\NormalTok{, }\StringTok{\textquotesingle{}c\textquotesingle{}}\NormalTok{: }\DecValTok{1}\NormalTok{\}), Vec(D, \{}\StringTok{\textquotesingle{}b\textquotesingle{}}\NormalTok{: }\DecValTok{1}\NormalTok{\})]}
\BuiltInTok{print}\NormalTok{(GF2\_span(D, L))}

\NormalTok{L }\OperatorTok{=}\NormalTok{ []}
\BuiltInTok{print}\NormalTok{(GF2\_span(D, L))}
\end{Highlighting}
\end{Shaded}

\begin{lstlisting}
[Vec({'a', 'c', 'b'},{'a': 0, 'c': 0, 'b': 0}), Vec({'a', 'c', 'b'},{'a': 0, 'c': 0, 'b': 1}), Vec({'a', 'c', 'b'},{'a': 1, 'c': 1, 'b': 0}), Vec({'a', 'c', 'b'},{'a': 1, 'c': 1, 'b': 1})]
[Vec({'a', 'c', 'b'},{})]
\end{lstlisting}

\newpage{}

\hypertarget{problem-3.8.4}{%
\subsection{Problem 3.8.4}\label{problem-3.8.4}}

Let \emph{a, b} be real numbers. Consider the equation \(z = ax + by\).
Prove that there are two 3-vectors \(\bm{v_1}, \bm{v_2}\) such that the
set of points \([x, y, z]\) satisfying the equation is exactly the set
of linear combinations of \(\bm{v_1}\) and \(\bm{v_2}\). (Hint: Specify
the vectors using formulas involving \(a, b\).)

To find the two 3-vectors \(\mathbf{v_1}\) and \(\mathbf{v_2}\) that
satisfy the equation \(z = ax + by\), we can start by considering the
cases where \(a\) and \(b\) are not both zero.

Case 1: \(a \neq 0, b \neq 0\)

In this case, we can choose
\(\mathbf{v_1} = \begin{bmatrix} 1 \ 0 \ a \end{bmatrix}\) and
\(\mathbf{v_2} = \begin{bmatrix} 0 \ 1 \ b \end{bmatrix}\). Then, any
point \(\begin{bmatrix} x \ y \ z \end{bmatrix}\) that satisfies the
equation \(z = ax + by\) can be written as a linear combination of
\(\mathbf{v_1}\) and \(\mathbf{v_2}\):

\(\begin{bmatrix} x \ y \ z \end{bmatrix} = x\begin{bmatrix} 1 \ 0 \ a \end{bmatrix} + y\begin{bmatrix} 0 \ 1 \ b \end{bmatrix} = \begin{bmatrix} x \ 0 \ ax \end{bmatrix} + \begin{bmatrix} 0 \ y \ by \end{bmatrix} = \begin{bmatrix} x \ y \ ax + by \end{bmatrix} = x\mathbf{v_1} + y\mathbf{v_2}\)

Conversely, any linear combination of \(\mathbf{v_1}\) and
\(\mathbf{v_2}\) can be written in the form
\(x\begin{bmatrix} 1 \ 0 \ a \end{bmatrix} + y\begin{bmatrix} 0 \ 1 \ b \end{bmatrix}\),
which satisfies the equation \(z = ax + by\).

Case 2: \(a = 0, b \neq 0\)

In this case, we can choose
\(\mathbf{v_1} = \begin{bmatrix} 0 \ 1 \ b \end{bmatrix}\) and
\(\mathbf{v_2} = \begin{bmatrix} 1 \ 0 \ 0 \end{bmatrix}\). Then, any
point \(\begin{bmatrix} x \ y \ z \end{bmatrix}\) that satisfies the
equation \(z = ax + by\) can be written as a linear combination of
\(\mathbf{v_1}\) and \(\mathbf{v_2}\):

\(\begin{bmatrix} x \ y \ z \end{bmatrix} = x\begin{bmatrix} 0 \ 1 \ b \end{bmatrix} + z\begin{bmatrix} 1 \ 0 \ 0 \end{bmatrix} = \begin{bmatrix} 0 \ x \ bx \end{bmatrix} + \begin{bmatrix} z \ 0 \ 0 \end{bmatrix} = \begin{bmatrix} z \ x \ bx \end{bmatrix} = z\mathbf{v_1} + x\mathbf{v_2}\)

Conversely, any linear combination of \(\mathbf{v_1}\) and
\(\mathbf{v_2}\) can be written in the form
\(x\begin{bmatrix} 1 \ 0 \ 0 \end{bmatrix} + z\begin{bmatrix} 0 \ 1 \ b \end{bmatrix} = \begin{bmatrix} x \ y \ bx \end{bmatrix}\),
which satisfies the equation \(z = ax + by\).

Case 3: \(a \neq 0, b = 0\)

This case is similar to Case 2, but with
\(\mathbf{v_1} = \begin{bmatrix} 1 \ 0 \ a \end{bmatrix}\) and
\(\mathbf{v_2} = \begin{bmatrix} 0 \ 1 \ 0 \end{bmatrix}\):

\(\begin{bmatrix} x \ y \ z \end{bmatrix} = x\begin{bmatrix} 1 \ 0 \ a \end{bmatrix} + z\begin{bmatrix} 0 \ 1 \ 0 \end{bmatrix} = \begin{bmatrix} x \ 0 \ ax \end{bmatrix} + \begin{bmatrix} 0 \ y \ 0 \end{bmatrix} = \begin{bmatrix} x \ y \ ax \end{bmatrix} = x\mathbf{v_1} + y\mathbf{v_2}\)

Again, any linear combination of \(\mathbf{v_1}\) and \(\mathbf{v_2}\)
can be written in the form
\(x\begin{bmatrix} 1 \ 0 \ a \end{bmatrix} + y\begin{bmatrix} 0 \ 1 \ 0 \end{bmatrix} = \begin{bmatrix} x \ y \ ax \end{bmatrix}\),
which satisfies the equation \(z = ax + by\).

Therefore, in all three cases, we have found two 3-vectors
\(\mathbf{v_1}\) and \(\mathbf{v_2}\) such that the set of points
satisfying the equation \(z = ax + by\) is exactly the set of linear
combinations of \(\mathbf{v_1}\) and \(\mathbf{v_2}\).

\newpage{}

\hypertarget{problem-3.8.5}{%
\subsection{Problem 3.8.5}\label{problem-3.8.5}}

Let \emph{a, b, c} be real numbers. Consider the equation
\(z = ax + by + c\). Prove that there are three 3-vectors
\(\bm{v_0}, \bm{v_1}, \bm{v_2}\) such that the set of points
\([x, y, z]\) satisfying the equation is exactly \begin{equation*}
\{\bm{v_0}+\alpha_1 \bm{v_1} + \alpha_2 \bm{v_2} \ : \ \alpha_1 \in \mathbb{R}, \alpha_2 \in \mathbb{R} \}
\end{equation*} (Hint: Specify the vectors using formulas involving
\(a, b, c\).)

Let \(v_0 = \begin{bmatrix}0\ 0\ c\end{bmatrix}, v_1 = \begin{bmatrix}1\ 0\ a\end{bmatrix}, v_2 = \begin{bmatrix}0\ 1\ b\end{bmatrix}\).
Now, let \([x,y,z]\) be a point that satisfies the equation
\(z = ax + by + c\). Then, we can write \([x,y,z]\) as: \\
\begin{align*}
  [x,y,z] &= \begin{bmatrix}x\ y\ ax+by+c\end{bmatrix} \\
  &= \begin{bmatrix}0\ 0\ c\end{bmatrix} + x\begin{bmatrix}1\ 0\ a\end{bmatrix} + y\begin{bmatrix}0\ 1\ b\end{bmatrix} \\
  &= v_0 + x v_1 + y v_2. \\
\end{align*}
Conversely, suppose $[x,y,z]$ is a point of the form $v_0 + x v_1 + y v_2$ for some $x,y \in \mathbb{R}$. Then, we have: \\
\begin{align*}
  [x,y,z] &= \begin{bmatrix}0\ 0\ c\end{bmatrix} + x\begin{bmatrix}1\ 0\ a\end{bmatrix} + y\begin{bmatrix}0\ 1\ b\end{bmatrix} \\
  &= \begin{bmatrix}x\ y\ ax+by+c\end{bmatrix}.
\end{align*}
Therefore, $z = ax + by + c$, and we see that every point of the form $v_0 + x v_1 + y v_2$ satisfies the equation.
\end{document}