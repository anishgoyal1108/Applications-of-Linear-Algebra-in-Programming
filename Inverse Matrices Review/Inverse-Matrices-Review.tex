% Options for packages loaded elsewhere
\PassOptionsToPackage{unicode}{hyperref}
\PassOptionsToPackage{hyphens}{url}
\PassOptionsToPackage{dvipsnames,svgnames,x11names}{xcolor}
%
\documentclass[
  letterpaper,
  DIV=11,
  numbers=noendperiod]{scrartcl}

\usepackage{amsmath,amssymb}
\usepackage{lmodern}
\usepackage{iftex}
\ifPDFTeX
  \usepackage[T1]{fontenc}
  \usepackage[utf8]{inputenc}
  \usepackage{textcomp} % provide euro and other symbols
\else % if luatex or xetex
  \usepackage{unicode-math}
  \defaultfontfeatures{Scale=MatchLowercase}
  \defaultfontfeatures[\rmfamily]{Ligatures=TeX,Scale=1}
\fi
% Use upquote if available, for straight quotes in verbatim environments
\IfFileExists{upquote.sty}{\usepackage{upquote}}{}
\IfFileExists{microtype.sty}{% use microtype if available
  \usepackage[]{microtype}
  \UseMicrotypeSet[protrusion]{basicmath} % disable protrusion for tt fonts
}{}
\makeatletter
\@ifundefined{KOMAClassName}{% if non-KOMA class
  \IfFileExists{parskip.sty}{%
    \usepackage{parskip}
  }{% else
    \setlength{\parindent}{0pt}
    \setlength{\parskip}{6pt plus 2pt minus 1pt}}
}{% if KOMA class
  \KOMAoptions{parskip=half}}
\makeatother
\usepackage{xcolor}
\setlength{\emergencystretch}{3em} % prevent overfull lines
\setcounter{secnumdepth}{-\maxdimen} % remove section numbering
% Make \paragraph and \subparagraph free-standing
\ifx\paragraph\undefined\else
  \let\oldparagraph\paragraph
  \renewcommand{\paragraph}[1]{\oldparagraph{#1}\mbox{}}
\fi
\ifx\subparagraph\undefined\else
  \let\oldsubparagraph\subparagraph
  \renewcommand{\subparagraph}[1]{\oldsubparagraph{#1}\mbox{}}
\fi


\providecommand{\tightlist}{%
  \setlength{\itemsep}{0pt}\setlength{\parskip}{0pt}}\usepackage{longtable,booktabs,array}
\usepackage{calc} % for calculating minipage widths
% Correct order of tables after \paragraph or \subparagraph
\usepackage{etoolbox}
\makeatletter
\patchcmd\longtable{\par}{\if@noskipsec\mbox{}\fi\par}{}{}
\makeatother
% Allow footnotes in longtable head/foot
\IfFileExists{footnotehyper.sty}{\usepackage{footnotehyper}}{\usepackage{footnote}}
\makesavenoteenv{longtable}
\usepackage{graphicx}
\makeatletter
\def\maxwidth{\ifdim\Gin@nat@width>\linewidth\linewidth\else\Gin@nat@width\fi}
\def\maxheight{\ifdim\Gin@nat@height>\textheight\textheight\else\Gin@nat@height\fi}
\makeatother
% Scale images if necessary, so that they will not overflow the page
% margins by default, and it is still possible to overwrite the defaults
% using explicit options in \includegraphics[width, height, ...]{}
\setkeys{Gin}{width=\maxwidth,height=\maxheight,keepaspectratio}
% Set default figure placement to htbp
\makeatletter
\def\fps@figure{htbp}
\makeatother

\usepackage{amsmath, xparse}
\usepackage{fancyvrb, fvextra}
\usepackage{bm}
\usepackage{svg}
\usepackage{listings}
\usepackage{sectsty}
\usepackage{xifthen, xparse}
\subsubsectionfont{\centering}
\DefineVerbatimEnvironment{Highlighting}{Verbatim}{breaklines,commandchars=\\\{\}}
\lstset{basicstyle=\ttfamily\footnotesize,breaklines=true}
\newcommand\rowop[1]{\scriptstyle\smash{\xrightarrow[\vphantom{#1}]{\mkern-4mu#1\mkern-4mu}}}
\DeclareDocumentCommand\converttorows%
{>{\SplitList{,}}m}%
{\ProcessList{#1}{\converttorow}}
\NewDocumentCommand{\converttorow}{m}
{\ifthenelse{\isempty{#1}}{}{\rowop{#1}}\\}

\DeclareDocumentCommand \rowops{m}
{\;
\begin{matrix}
\converttorows {#1}
\end{matrix}
\; }
\KOMAoption{captions}{tableheading}
\makeatletter
\makeatother
\makeatletter
\makeatother
\makeatletter
\@ifpackageloaded{caption}{}{\usepackage{caption}}
\AtBeginDocument{%
\ifdefined\contentsname
  \renewcommand*\contentsname{Table of contents}
\else
  \newcommand\contentsname{Table of contents}
\fi
\ifdefined\listfigurename
  \renewcommand*\listfigurename{List of Figures}
\else
  \newcommand\listfigurename{List of Figures}
\fi
\ifdefined\listtablename
  \renewcommand*\listtablename{List of Tables}
\else
  \newcommand\listtablename{List of Tables}
\fi
\ifdefined\figurename
  \renewcommand*\figurename{Figure}
\else
  \newcommand\figurename{Figure}
\fi
\ifdefined\tablename
  \renewcommand*\tablename{Table}
\else
  \newcommand\tablename{Table}
\fi
}
\@ifpackageloaded{float}{}{\usepackage{float}}
\floatstyle{ruled}
\@ifundefined{c@chapter}{\newfloat{codelisting}{h}{lop}}{\newfloat{codelisting}{h}{lop}[chapter]}
\floatname{codelisting}{Listing}
\newcommand*\listoflistings{\listof{codelisting}{List of Listings}}
\makeatother
\makeatletter
\@ifpackageloaded{caption}{}{\usepackage{caption}}
\@ifpackageloaded{subcaption}{}{\usepackage{subcaption}}
\makeatother
\makeatletter
\@ifpackageloaded{tcolorbox}{}{\usepackage[many]{tcolorbox}}
\makeatother
\makeatletter
\@ifundefined{shadecolor}{\definecolor{shadecolor}{rgb}{.97, .97, .97}}
\makeatother
\makeatletter
\makeatother
\ifLuaTeX
  \usepackage{selnolig}  % disable illegal ligatures
\fi
\IfFileExists{bookmark.sty}{\usepackage{bookmark}}{\usepackage{hyperref}}
\IfFileExists{xurl.sty}{\usepackage{xurl}}{} % add URL line breaks if available
\urlstyle{same} % disable monospaced font for URLs
\hypersetup{
  colorlinks=true,
  linkcolor={blue},
  filecolor={Maroon},
  citecolor={Blue},
  urlcolor={Blue},
  pdfcreator={LaTeX via pandoc}}

\author{}
\date{}

\begin{document}
\begin{titlepage}

    \newcommand{\HRule}{\rule{\linewidth}{0.5mm}}
    
    \center
    
    \textsc{\LARGE GSMST }\\[0.3cm]
    \textsc{\Large Applications of Linear Algebra }\\[0.3cm]
    \textsc{\Large in Programming}\\[0.5cm]
    
    \HRule \\[0.4cm]
    { \huge \bfseries Chapter 3 Assignment}\\[0.03cm]
    \HRule \\[1.5cm]
    
    \begin{minipage}{0.4\textwidth}
    \begin{flushleft} \large
    \emph{Submitted By:}\\
    Anish Goyal \\4th Period
    \end{flushleft}
    \end{minipage}
    ~
    \begin{minipage}{0.4\textwidth}
    \begin{flushright} \large
    \emph{Submitted To:} \\
    Mrs. Denise Stiffler\\Educator
    \end{flushright}
    \end{minipage}\\[1cm]
    
    {\large April 11, 2023}\\[1cm]
    
    \includegraphics{logo.png}\\[1cm]
    \vfill
    \end{titlepage}
\newpage

\ifdefined\Shaded\renewenvironment{Shaded}{\begin{tcolorbox}[sharp corners, borderline west={3pt}{0pt}{shadecolor}, boxrule=0pt, interior hidden, enhanced, breakable, frame hidden]}{\end{tcolorbox}}\fi

\renewcommand*\contentsname{Table of contents}
{
\hypersetup{linkcolor=}
\setcounter{tocdepth}{4}
\tableofcontents
}
\newpage{}

\hypertarget{find-the-inverse-of-each-matrix.-you-must-use-each-method-from-class-augmentedcofactoring-and-linear-row-reduction-with-identity-matrix}{%
\subsection{1. Find the inverse of each matrix. You must use each method
from class (Augmented/Cofactoring and linear row reduction with identity
matrix):}\label{find-the-inverse-of-each-matrix.-you-must-use-each-method-from-class-augmentedcofactoring-and-linear-row-reduction-with-identity-matrix}}

\hypertarget{gauss-jordan-elimination-method}{%
\subsubsection{Gauss-Jordan Elimination
Method}\label{gauss-jordan-elimination-method}}

\begin{align*}
A = \begin{bmatrix}
-2 & -6 & 1 \\
-3 & -2 & -5 \\
-5 & -4 & -1
\end{bmatrix}
\end{align*}

\centering \textcolor[RGB]{0,0,0}{\rule{\linewidth}{0.6pt}}

\begin{align*}
[A \ \ I] = \left[\begin{array}{ccc|ccc}
-2 & -6 & 1 & 1 & 0 & 0 \\
-3 & -2 & -5 & 0 & 1 & 0 \\
-5 & -4 & -1 & 0 & 0 & 1
\end{array}\right]
\rowops{-\frac{1}{2}R_1,,}
\left[\begin{array}{ccc|ccc}
1 & 3 & -\frac{1}{2} & -\frac{1}{2} & 0 & 0 \\
-3 & -2 & -5 & 0 & 1 & 0 \\
-5 & -4 & -1 & 0 & 0 & 1
\end{array}\right] &\\
\rowops{,3R_1+2R_2,5R_1+R_3}
\left[\begin{array}{ccc|ccc}
1 & 3 & -\frac{1}{2} & -\frac{1}{2} & 0 & 0 \\
0 & 7 & -\frac{13}{2} & -\frac{3}{2} & 1 & 0 \\
0 & 11 & -\frac{7}{2} & -\frac{5}{2} & 0 & 1
\end{array}\right] \rowops{,\frac{1}{7}R_2,} \left[\begin{array}{ccc|ccc}
  1 & 3 & -\frac{1}{2} & -\frac{1}{2} & 0 & 0 \\
  0 & 1 & -\frac{13}{14} & -\frac{3}{14} & \frac{1}{7} & 0 \\
  0 & 11 & -\frac{7}{2} & -\frac{5}{2} & 0 & 1
\end{array}\right] &\\
\rowops{R_1-3R_2,,} \left[\begin{array}{ccc|ccc}
  1 & 0 & \frac{16}{7} & \frac{1}{7} & -\frac{3}{7} & 0 \\
  0 & 1 & -\frac{13}{14} & -\frac{3}{14} & \frac{1}{7} & 0 \\
  0 & 11 & -\frac{7}{2} & -\frac{5}{2} & 0 & 1
\end{array}\right] \rowops{,,R_3-11R_2} \left[\begin{array}{ccc|ccc}
  1 & 0 & \frac{16}{7} & \frac{1}{7} & -\frac{3}{7} & 0 \\
  0 & 1 & -\frac{13}{14} & -\frac{3}{14} & \frac{1}{7} & 0 \\
  0 & 0 & \frac{47}{7} & -\frac{1}{7} & -\frac{11}{7} & 1
\end{array}\right] &\\
\rowops{,,\frac{4}{47}R_3} \left[\begin{array}{ccc|ccc}
  1 & 0 & \frac{16}{7} & \frac{1}{7} & -\frac{3}{7} & 0 \\
  0 & 1 & -\frac{13}{14} & -\frac{3}{14} & \frac{1}{7} & 0 \\
  0 & 0 & 1 & -\frac{1}{47} & -\frac{11}{47} & \frac{7}{47}
\end{array}\right]
\rowops{R_1 - \frac{16}{7}R_3,,} \left[\begin{array}{ccc|ccc}
    1 & 0 & 0 & \frac{9}{47} & \frac{5}{47} & -\frac{16}{47} \\
    0 & 1 & -\frac{13}{14} & -\frac{3}{14} & \frac{1}{7} & 0 \\
    0 & 0 & 1 & -\frac{1}{47} & -\frac{11}{47} & \frac{7}{47}
\end{array}\right] &\\
\rowops{,\frac{13}{14}R_1+R_2,}
\left[\begin{array}{ccc|ccc}
    1 & 0 & 0 & \frac{9}{47} & \frac{5}{47} & -\frac{16}{47} \\
    0 & 1 & 0 & -\frac{11}{47} & -\frac{7}{94} & \frac{13}{94} \\
    0 & 0 & 1 & -\frac{1}{47} & -\frac{11}{47} & \frac{7}{47}
\end{array}\right] &\\
A^{-1} = \begin{bmatrix}
\frac{9}{47} & \frac{5}{47} & -\frac{16}{47} \\
-\frac{11}{47} & -\frac{7}{94} & \frac{13}{94} \\
-\frac{1}{47} & -\frac{11}{47} & \frac{7}{47}
\end{bmatrix} &\\
\end{align*}

\newpage{}

\hypertarget{adjoint-method}{%
\subsubsection{Adjoint Method}\label{adjoint-method}}

\begin{gather*}
B = \begin{bmatrix}
3 & 4 & 5 \\
-2 & 3 & 3 \\
-1 & 2 & -5
\end{bmatrix} \\
\end{gather*}

\centering \textcolor[RGB]{0,0,0}{\rule{\linewidth}{0.6pt}}

\begin{align*}
\det B &= \begin{vmatrix}
3 & 4 & 5 \\
-2 & 3 & 3 \\
-1 & 2 & -5
\end{vmatrix} \\
&=3\begin{vmatrix}
3 & 3 \\
2 & -5
\end{vmatrix} + 4\begin{vmatrix}
2 & 3 \\
1 & -5
\end{vmatrix} + 5\begin{vmatrix}
-2 & 3 \\
-1 & 2
\end{vmatrix} \\
&= 3([3 \cdot -5] - [3 \cdot 2]) + 4([2 \cdot -5] - [3 \cdot 1]) + 5([-2 \cdot 2] - [3 \cdot -1]) \\
&= 3(-15 - 6) + 4(-10 - 3) + 5(-4 + 3) \\
&= 3(-21) + 4(-13) + 5(-1) \\
&= -63 - 52 - 5 \\
&= -120
\end{align*}

\begin{align*}
\mathrm{cof}(B_{11}) &= \begin{vmatrix}
3 & 3 \\
2 & -5
\end{vmatrix} &= -21 \qquad
\mathrm{cof}(B_{12}) &= -\begin{vmatrix}
-2 & 3 \\
-1 & -5
\end{vmatrix} &= -13 \qquad
\mathrm{cof}(B_{13}) &= \begin{vmatrix}
-2 & 3 \\
-1 & 2
\end{vmatrix} &= -1 \\
\mathrm{cof}(B_{21}) &= -\begin{vmatrix}
4 & 5 \\
2 & -5
\end{vmatrix} &= 30 \qquad
\mathrm{cof}(B_{22}) &= \begin{vmatrix}
3 & 5 \\
-1 & -5
\end{vmatrix} &= -10 \qquad
\mathrm{cof}(B_{23}) &= -\begin{vmatrix}
3 & 4 \\
-1 & 2
\end{vmatrix} &= -10 \\
\mathrm{cof}(B_{31}) &= \begin{vmatrix}
4 & 5 \\
3 & 3
\end{vmatrix} &= -3 \qquad
\mathrm{cof}(B_{32}) &= -\begin{vmatrix}
3 & 5 \\
-2 & 3
\end{vmatrix} &= -19 \qquad
\mathrm{cof}(B_{33}) &= \begin{vmatrix}
3 & 4 \\
-2 & 3 
\end{vmatrix} &= 17
\end{align*}

\begin{align*}
[\mathrm{cof}(B_{ij})] = \begin{bmatrix}
-21 & -13 & -1 \\
30 & -10 & -10 \\
-3 & -19 & 17
\end{bmatrix} \\
\mathrm{Adj} \ B = [\mathrm{cof}(B_{ij})]^\intercal = \begin{bmatrix}
-21 & 30 & -3 \\
-13 & -10 & -19 \\
-1 & -10 & 17
\end{bmatrix} \\
\end{align*}

\newpage{}

\begin{align*}
B^{-1} &= \frac{\mathrm{Adj } \ B}{\det B} \\
&= \left(-\frac{1}{120}\right)\begin{bmatrix}
-21 & 30 & -3 \\
-13 & -10 & -19 \\
-1 & -10 & 17
\end{bmatrix} \\  
&= \begin{bmatrix}
-21 \cdot \left(-\frac{1}{120}\right) & 30 \cdot \left(-\frac{1}{120}\right) & -3 \cdot \left(-\frac{1}{120}\right) \\
-13 \cdot \left(-\frac{1}{120}\right) & -10 \cdot \left(-\frac{1}{120}\right) & -19 \cdot \left(-\frac{1}{120}\right) \\
-1 \cdot \left(-\frac{1}{120}\right) & -10 \cdot \left(-\frac{1}{120}\right) & 17 \cdot \left(-\frac{1}{120}\right)
\end{bmatrix} \\
&= \begin{bmatrix}
\frac{-21}{-120} & \frac{30}{-120} & \frac{-3}{-120} \\
\frac{-13}{-120} & \frac{-10}{-120} & \frac{-19}{-120} \\
\frac{-1}{-120} & \frac{-10}{-120} & \frac{17}{-120}
\end{bmatrix} \\
&= \begin{bmatrix}
\frac{7}{40} & -\frac{1}{4} & \frac{1}{40} \\
\frac{13}{120} & \frac{1}{12} & \frac{19}{120} \\
\frac{1}{120} & \frac{1}{12} & -\frac{17}{120}
\end{bmatrix}
\end{align*}

\newpage{}

\hypertarget{for-which-three-numbers-c-is-this-matrix-not-invertible-and-why-not}{%
\subsection{\texorpdfstring{2. For which three numbers \(c\) is this
matrix not invertible, and why
not?}{2. For which three numbers c is this matrix not invertible, and why not?}}\label{for-which-three-numbers-c-is-this-matrix-not-invertible-and-why-not}}

\[
A = \begin{bmatrix}
2 & c & c \\
c & c & c \\
8 & 7 & c
\end{bmatrix}
\]

\centering \textcolor[RGB]{0,0,0}{\rule{\linewidth}{0.6pt}}

The three numbers of \(c\) for which \(A\) is not invertible are
\textbf{0, 7, and 2}, as those values of \(c\) would make \(A\) linearly
dependent.

If \(c\) was 0, then \(A\) would be linearly dependent because it
contained a column/row of zeros.

If \(c\) was 7, then \(A\) would be linearly dependent because it would
contain a column of 7s.

If \(c\) was 2, then \(A\) would be linearly dependent because it would
contain a row of 2s.

\newpage{}

\hypertarget{prove-that-a-is-invertible-if-a-ne-0-and-a-ne-b-find-the-pivots-or-a-1}{%
\subsection{\texorpdfstring{3. Prove that \(A\) is invertible if
\(a \ne 0\) and \(a \ne b\) (find the pivots or
\(A^{-1}\)):}{3. Prove that A is invertible if a \textbackslash ne 0 and a \textbackslash ne b (find the pivots or A\^{}\{-1\}):}}\label{prove-that-a-is-invertible-if-a-ne-0-and-a-ne-b-find-the-pivots-or-a-1}}

\[
A = \begin{bmatrix}
a & b & b \\
a & a & b \\
a & a & a
\end{bmatrix} \text{.}
\]

\centering \textcolor[RGB]{0,0,0}{\rule{\linewidth}{0.6pt}}

Let's perform a row reduction of the given matrix \(A\):

\begin{gather*}
\left(\begin{matrix}
a & b & b \\
a & a & b \\
a & a & a
\end{matrix}\right) \sim \left(\begin{matrix}
a & b & b \\
0 & a-b & 0 \\
0 & a-b & a-b
\end{matrix}\right) \sim \left(\begin{matrix}
a & b & b \\
0 & a-b & 0 \\
0 & 0 & a-b
\end{matrix}\right)
\end{gather*}

Therefore, \(A\) is invertible since \(a \ne 0\) and
\(a \ne b \therefore a-b \ne 0\), which means all the pivots are
non-zero in fully reduced row-echelon form.

\newpage{}

\hypertarget{this-matrix-has-a-remarkable-inverse.-find-a-1-by-elimination-on-a-i.-extend-to-a-5-by-5-alternating-matrix-and-guess-its-inverse-then-multiply-to-confirm.}{%
\subsection{\texorpdfstring{4. This matrix has a remarkable inverse.
Find \(A^{-1}\) by elimination on \([A \ \ I]\). Extend to a 5 by 5
``alternating matrix'' and guess its inverse; then multiply to
confirm.}{4. This matrix has a remarkable inverse. Find A\^{}\{-1\} by elimination on {[}A \textbackslash{} \textbackslash{} I{]}. Extend to a 5 by 5 ``alternating matrix'' and guess its inverse; then multiply to confirm.}}\label{this-matrix-has-a-remarkable-inverse.-find-a-1-by-elimination-on-a-i.-extend-to-a-5-by-5-alternating-matrix-and-guess-its-inverse-then-multiply-to-confirm.}}

\[
\mathrm{Invert} \  A = \begin{bmatrix}
1 & -1 & 1 & -1 \\
0 & 1 & -1 & 1 \\
0 & 0 & 1 & -1 \\
0 & 0 & 0 & 1
\end{bmatrix} \ \text{and solve }Ax = (1, 1, 1, 1).
\]

\centering \textcolor[RGB]{0,0,0}{\rule{\linewidth}{0.6pt}}

\begin{align*}
[A \ \ I] = \left[\begin{array}{cccc|cccc}
1 & -1 & 1 & -1 & 1 & 0 & 0 & 0 \\
0 & 1 & -1 & 1 & 0 & 1 & 0 & 0 \\
0 & 0 & 1 & -1 & 0 & 0 & 1 & 0 \\
0 & 0 & 0 & 1 & 0 & 0 & 0 & 1
\end{array}\right] \rowops{R_2+R1,,,} \left[\begin{array}{cccc|cccc}
1 & 0 & 0 & 0 & 1 & 1 & 0 & 0 \\
0 & 1 & -1 & 1 & 0 & 1 & 0 & 0 \\
0 & 0 & 1 & -1 & 0 & 0 & 1 & 0 \\
0 & 0 & 0 & 1 & 0 & 0 & 0 & 1
\end{array}\right] &\\
\rowops{,R_3+R_2,,} \left[\begin{array}{cccc|cccc}
1 & 0 & 0 & 0 & 1 & 1 & 0 & 0 \\
0 & 1 & 0 & 0 & 0 & 1 & 1 & 0 \\
0 & 0 & 1 & -1 & 0 & 0 & 1 & 0 \\
0 & 0 & 0 & 1 & 0 & 0 & 0 & 1
\end{array}\right] \rowops{,,R_4+R_3,} \left[\begin{array}{cccc|cccc}
1 & 0 & 0 & 0 & 1 & 1 & 0 & 0 \\
0 & 1 & 0 & 0 & 0 & 1 & 1 & 0 \\
0 & 0 & 1 & 0 & 0 & 0 & 1 & 1 \\
0 & 0 & 0 & 1 & 0 & 0 & 0 & 1
\end{array}\right] &\\
A^{-1} = \begin{bmatrix}
1 & 1 & 0 & 0 \\
0 & 1 & 1 & 0 \\
0 & 0 & 1 & 1 \\
0 & 0 & 0 & 1
\end{bmatrix} &\\
\end{align*}

Now, let's extend \(A\) to be a 5x5 alternating matrix, which we'll
denote as \(B\):

\[
B = \begin{bmatrix}
1 & -1 & 1 & -1 & 1 \\
0 & 1 & -1 & 1 & -1 \\
0 & 0 & 1 & -1 & 1 \\
0 & 0 & 0 & 1 & -1 \\
0 & 0 & 0 & 0 & 1
\end{bmatrix}
\]

If we wanted to calculate \(B^{-1}\), we would have to reduce it just
like we did to \(A\) above\ldots{} but the pattern here is pretty
obvious, since \(A^{-1}\) follows a ``snake'' pattern downward:

\[
B^{-1} \text{ (predicted) } = \begin{bmatrix}
1 & 1 & 0 & 0 & 0 \\
0 & 1 & 1 & 0 & 0 \\
0 & 0 & 1 & 1 & 0 \\
0 & 0 & 0 & 1 & 1 \\
0 & 0 & 0 & 0 & 1
\end{bmatrix}
\]

\newpage{}

Let's actually calculate it:

\begin{align*}
[B \ \ I] = \left[\begin{array}{ccccc|ccccc}
1 & -1 & 1 & -1 & 1 & 1 & 0 & 0 & 0 & 0 \\
0 & 1 & -1 & 1 & -1 & 0 & 1 & 0 & 0 & 0 \\
0 & 0 & 1 & -1 & 1 & 0 & 0 & 1 & 0 & 0 \\
0 & 0 & 0 & 1 & -1 & 0 & 0 & 0 & 1 & 0 \\
0 & 0 & 0 & 0 & 1 & 0 & 0 & 0 & 0 & 1
\end{array}\right] \rowops{R_2+R1,,,,} \left[\begin{array}{ccccc|ccccc}
1 & 0 & 0 & 0 & 0 & 1 & 1 & 0 & 0 & 0 \\
0 & 1 & -1 & 1 & -1 & 0 & 1 & 0 & 0 & 0 \\
0 & 0 & 1 & -1 & 1 & 0 & 0 & 1 & 0 & 0 \\
0 & 0 & 0 & 1 & -1 & 0 & 0 & 0 & 1 & 0 \\
0 & 0 & 0 & 0 & 1 & 0 & 0 & 0 & 0 & 1
\end{array}\right] &\\
\rowops{,R3+R2,,,} \left[\begin{array}{ccccc|ccccc}
1 & 0 & 0 & 0 & 0 & 1 & 1 & 0 & 0 & 0 \\
0 & 1 & 0 & 0 & 0 & 0 & 1 & 1 & 0 & 0 \\
0 & 0 & 1 & -1 & 1 & 0 & 0 & 1 & 0 & 0 \\
0 & 0 & 0 & 1 & -1 & 0 & 0 & 0 & 1 & 0 \\
0 & 0 & 0 & 0 & 1 & 0 & 0 & 0 & 0 & 1
\end{array}\right] \rowops{,,R_4+R_3,,} \left[\begin{array}{ccccc|ccccc}
1 & 0 & 0 & 0 & 0 & 1 & 1 & 0 & 0 & 0 \\
0 & 1 & 0 & 0 & 0 & 0 & 1 & 1 & 0 & 0 \\
0 & 0 & 1 & 0 & 0 & 0 & 0 & 1 & 1 & 0 \\
0 & 0 & 0 & 1 & -1 & 0 & 0 & 0 & 1 & 0 \\
0 & 0 & 0 & 0 & 1 & 0 & 0 & 0 & 0 & 1
\end{array}\right] &\\
\rowops{,,,R_5+R4,} \left[\begin{array}{ccccc|ccccc}
1 & 0 & 0 & 0 & 0 & 1 & 1 & 0 & 0 & 0 \\
0 & 1 & 0 & 0 & 0 & 0 & 1 & 1 & 0 & 0 \\
0 & 0 & 1 & 0 & 0 & 0 & 0 & 1 & 1 & 0 \\
0 & 0 & 0 & 1 & 0 & 0 & 0 & 0 & 1 & 1 \\
0 & 0 & 0 & 0 & 1 & 0 & 0 & 0 & 0 & 1
\end{array}\right] &\\
B^{-1} = \begin{bmatrix}
1 & 1 & 0 & 0 & 0 \\
0 & 1 & 1 & 0 & 0 \\
0 & 0 & 1 & 1 & 0 \\
0 & 0 & 0 & 1 & 1 \\
0 & 0 & 0 & 0 & 1 
\end{bmatrix} &\\
\end{align*}

Now, let's confirm \(B^{-1} \cdot B\) yields the identity matrix:

\[
B^{-1} \cdot B = \begin{bmatrix}
1 & -1 & 1 & -1 & 1 \\
0 & 1 & -1 & 1 & -1 \\
0 & 0 & 1 & -1 & 1 \\
0 & 0 & 0 & 1 & -1 \\
0 & 0 & 0 & 0 & 1
\end{bmatrix} \begin{bmatrix}
1 & 1 & 0 & 0 & 0 \\
0 & 1 & 1 & 0 & 0 \\
0 & 0 & 1 & 1 & 0 \\
0 & 0 & 0 & 1 & 1 \\
0 & 0 & 0 & 0 & 1 
\end{bmatrix} = \begin{bmatrix}
C_{11} & C_{12} & \cdots & C_{15}\\
C_{21} & C_{22} & \cdots & C_{25}\\ 
\vdots & \vdots & \ddots & \vdots\\ 
C_{51} & C_{52} & \cdots & C_{55} 
\end{bmatrix}
\] \[
C_{ij}= B_{i1} (B^{-1})_{1j} + B_{i2} (B^{-1})_{2j} +\cdots+ B_{in} + (B^{-1})_{nj} = \sum_{k=1}^n B_{ik}(B^{-1})_{kj}
\] \[
\stackrel{\checkmark}{=} \begin{bmatrix}
1 & 0 & 0 & 0 & 0 \\
0 & 1 & 0 & 0 & 0 \\
0 & 0 & 1 & 0 & 0 \\
0 & 0 & 0 & 1 & 0 \\
0 & 0 & 0 & 0 & 1
\end{bmatrix}
\]

\newpage{}

\hypertarget{a-is-a-4x4-matrix-with-1s-on-the-diagonal-and--a--b--c-on-the-diagonal-above.-find-the-inverse-of-this-bidiagonal-matrix.}{%
\subsection{\texorpdfstring{5. \(A\) is a 4x4 matrix with 1's on the
diagonal and \(-a, -b, -c\) on the diagonal above. Find the inverse of
this bidiagonal
matrix.}{5. A is a 4x4 matrix with 1's on the diagonal and -a, -b, -c on the diagonal above. Find the inverse of this bidiagonal matrix.}}\label{a-is-a-4x4-matrix-with-1s-on-the-diagonal-and--a--b--c-on-the-diagonal-above.-find-the-inverse-of-this-bidiagonal-matrix.}}

\centering \textcolor[RGB]{0,0,0}{\rule{\linewidth}{0.6pt}}

\begin{align*}
[A \ \ I] = \left[\begin{array}{cccc|cccc}
1 & -a & 0 & 0 & 1 & 0 & 0 & 0 \\
0 & 1 & -b & 0 & 0 & 1 & 0 & 0 \\
0 & 0 & 1 & -c & 0 & 0 & 1 & 0 \\
0 & 0 & 0 & 1 & 0 & 0 & 0 & 1
\end{array}\right]
\rowops{aR_2+R_1,,,}
\left[\begin{array}{cccc|cccc}
1 & 0 & -ab & 0 & 1 & a & 0 & 0 \\
0 & 1 & -b & 0 & 0 & 1 & 0 & 0 \\
0 & 0 & 1 & -c & 0 & 0 & 1 & 0 \\
0 & 0 & 0 & 1 & 0 & 0 & 0 & 1
\end{array}\right] &\\
\rowops{abR_3+R_1,,,}
\left[\begin{array}{cccc|cccc}
1 & 0 & 0 & -abc & 1 & a & ab & 0 \\
0 & 1 & -b & 0 & 0 & 1 & 0 & 0 \\
0 & 0 & 1 & -c & 0 & 0 & 1 & 0 \\
0 & 0 & 0 & 1 & 0 & 0 & 0 & 1
\end{array}\right]
\rowops{,bR_3+R_2,,}
\left[\begin{array}{cccc|cccc}
1 & 0 & 0 & -abc & 1 & a & ab & 0 \\
0 & 1 & 0 & -bc & 0 & 1 & b & 0 \\
0 & 0 & 1 & -c & 0 & 0 & 1 & 0 \\
0 & 0 & 0 & 1 & 0 & 0 & 0 & 1
\end{array}\right] &\\
\rowops{abcR_4+R_1,,,}
\left[\begin{array}{cccc|cccc}
1 & 0 & 0 & 0 & 1 & a & ab & abc \\
0 & 1 & 0 & -bc & 0 & 1 & b & 0 \\
0 & 0 & 1 & -c & 0 & 0 & 1 & 0 \\
0 & 0 & 0 & 1 & 0 & 0 & 0 & 1
\end{array}\right]
\rowops{,bcR_4+R_2,,}
\left[\begin{array}{cccc|cccc}
1 & 0 & 0 & 0 & 1 & a & ab & abc \\
0 & 1 & 0 & 0 & 0 & 1 & b & bc \\
0 & 0 & 1 & -c & 0 & 0 & 1 & 0 \\
0 & 0 & 0 & 1 & 0 & 0 & 0 & 1
\end{array}\right] &\\
\rowops{,,cR_4+R_3,}
\left[\begin{array}{cccc|cccc}
1 & 0 & 0 & 0 & 1 & a & ab & abc \\
0 & 1 & 0 & 0 & 0 & 1 & b & bc \\
0 & 0 & 1 & 0 & 0 & 0 & 1 & c \\
0 & 0 & 0 & 1 & 0 & 0 & 0 & 1
\end{array}\right] &\\
A^{-1} = \begin{bmatrix}
1 & a & ab & abc \\
0 & 1 & b & bc \\
0 & 0 & 1 & c \\
0 & 0 & 0 & 1
\end{bmatrix} &\\
\end{align*}



\end{document}
